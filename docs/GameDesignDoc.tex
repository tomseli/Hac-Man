\documentclass[11pt]{Article}
% \usepackage{fullpage}
\usepackage{graphicx}
\usepackage{amsmath}
\usepackage{pdfpages}
\usepackage{lipsum}
\usepackage{blindtext} 
\usepackage{fancyhdr}
\usepackage[colorinlistoftodos]{todonotes}


%syntax highlighting and code blocks
\usepackage{listings}
% \usepackage{minted}

\usepackage{tikz}
\usetikzlibrary{positioning,calc}

% Language setting
% Replace `english' with e.g. `spanish' to change the document language
% \usepackage[english]{babel}
\usepackage{multicol}
\usepackage{geometry}
\usepackage{caption}
\usepackage{float}
\usepackage[hidelinks]{hyperref}
\usepackage[backend=biber,style=apa,autocite=inline]{biblatex}
\DeclareLanguageMapping{english}{english-apa}
\setlength{\columnsep}{1cm}
\geometry{
        a4paper,
        total={170mm,257mm},
        left=20mm,
        top=20mm,
        }

\usepackage{listings}
\usepackage{xcolor}

%%%%%%%%%%%%%%%%%%%%%%%%%%%%%%%%%%%%%%%%%%%%%%%%%%%%%%%%%%%%%%%%%%%%%%%%%%%%%%%%%%%%%%%%%%%%%%%%
%%%%%%%%%%%%%%%%%%%%%%%%%%%%%%%%%%%%%%%%%%%%%%%%%%%%%%%%%%%%%%%%%%%%%%%%%%%%%%%%%%%%%%%%%%%%%%%%


\definecolor{codegreen}{rgb}{0,0.6,0}
\definecolor{codegray}{rgb}{0.5,0.5,0.5}
\definecolor{codepurple}{rgb}{0.58,0,0.82}
\definecolor{backcolour}{rgb}{0.95,0.95,0.92}

\lstdefinestyle{mystyle}{
    backgroundcolor=\color{backcolour},   
    commentstyle=\color{codegreen},
    keywordstyle=\color{blue},
    numberstyle=\tiny\color{codegray},
    stringstyle=\color{codepurple},
    basicstyle=\ttfamily\footnotesize,
    breakatwhitespace=false,         
    breaklines=true,                 
    captionpos=b,                    
    keepspaces=true,                 
    numbers=left,                    
    numbersep=5pt,                  
    showspaces=false,                
    showstringspaces=false,
    showtabs=false,                  
    tabsize=2
} 

%%%%%%%%%%%%%%%%%%%%%%%%%%%%%%%%%%%%%%%%%%%%%%%%%%%%%%%%%%%%%%%%%%%%%%%%%%%%%%%%%%%%%%%%%%%%%%%%
%%%%%%%%%%%%%%%%%%%%%%%%%%%%%%%%%%%%%%%%%%%%%%%%%%%%%%%%%%%%%%%%%%%%%%%%%%%%%%%%%%%%%%%%%%%%%%%%


% Useful packages
% \usepackage{amsmath}
% \usepackage{siunitx}
% \usepackage{wrapfig}
% \usepackage{float}
% \usepackage{graphicx}
% \usepackage{subcaption}
% \usepackage[colorlinks=true, allcolors=blue]{hyperref}
% \usepackage{xcolor}

\graphicspath{
    {img/}
}
\bibliography{./bib.bib}

\newenvironment{Figure}
  {\par\medskip\noindent\minipage{\linewidth}}
  {\endminipage\par\medskip}
\newcommand\C[1]\null

%remove if done, this extends the page to see comments
% \paperwidth=\dimexpr \paperwidth + 8cm\relax
% \oddsidemargin=\dimexpr\oddsidemargin + 3cm\relax
% \evensidemargin=\dimexpr\evensidemargin + 3cm\relax
% \marginparwidth=\dimexpr \marginparwidth + 3cm\relax


\title{Hac-man: Game design document}

\author{%
  \begin{tabular}{c c c}
    Mirko van de Hoef   &    Tom Selier\\
    \texttt{m.a.j.vandeHoef@students.uu.nl} & \texttt{t.b.j.selier@students.uu.nl} \\
    Student Number: 2176777 & Student Number: 5412498 
  \end{tabular}
}


\usepackage{eso-pic}
\usepackage{atbegshi}
\AtBeginShipoutFirst{\AddToShipoutPictureBG*{
    \begin{tikzpicture}[overlay, remember picture, outer sep=0pt, inner sep=1ex]
      \node [anchor=north west] at ($(current page.north)-(.5*\textwidth-1ex, \headheight)$)
      {\parbox{10cm}{UU University Utrecht\\ Functional Programming\\2024, Utrecht, Netherlands}};
    \end{tikzpicture}
}}




%%%%%%%%%%%%%%%%%%%%%%%%%%%%%%%%%%%%%%%%%%%%%%%%%%%%%%%%%%%%%%%%%%%%%%%%%%%%%%%%%%%%%%%%%%%%%%%%
%%%%%%%%%%%%%%%%%%%%%%%%%%%%%%%%%%%%%%%%%%%%%%%%%%%%%%%%%%%%%%%%%%%%%%%%%%%%%%%%%%%%%%%%%%%%%%%%
%beginning of document
\begin{document}
\lstset{style=mystyle}
\maketitle
% \begin{abstract}
% \lipsum[1]
% \end{abstract}
{\bf \textit{Keywords--}} game game gamin
%%%%%%%%%%%%%%%%%%%%%%%%%%%%%%%%%%%%%%%%%%%%%%%%%%%%%%%%%%%%%%%%%%%%%%%%%%%%%%%%%%%%%%%%%%%%%%%%
%%%%%%%%%%%%%%%%%%%%%%%%%%%%%%%%%%%%%%%%%%%%%%%%%%%%%%%%%%%%%%%%%%%%%%%%%%%%%%%%%%%%%%%%%%%%%%%%

%introduction
\section{Introduction} \label{ch:Introduction}
The objective of this design document is to present the design choices for the creation of Pac-Man, along with their respective justifications. The decision to implement Pac-man is based on on the expected freedom the game offers in its implementation. Furthermore, Pac-Man will allow us to implement some of the optional requirements in addition to the minimum requirements. Additionally, some variations on the fundemental gameplay are introduced to make the game more unique.


% \begin{itemize}
%   \item [\textbf{Player}]
% \end{itemize}

\section{Variations on the base game} \label{ch:twist}
In order to introduce a new learning opportunity and make the game more unique, a player adversary will be introduced. This will introduce, a co-op type, multiplayer aspect to the game. This feature will introduce a second set of player inputs which allows the control of one of the ghosts.


\section{Data types} \label{ch:dataTypes}
\begin{lstlisting}[language=Haskell]
  --haskell code
\end{lstlisting}


\section{Minumum Requirements} \label{ch:minumumRequirements}

\textbf{Player} The player can control Pac-Man, and the menus, using the WASD-keys on the keyboard.

\textbf{Enemies} The computer controls the red ghost, called Blinky. Blinky is the simplest of four ghost and tries to chase Pac-Man.

\textbf{Randomness} A ghost can exist in three states; Scatter, chase and frightened. When a ghost is frightened it will pick a random direction at each intersection.

\textbf{Animation} The amount of animations in Pac-Man is limited. The animations that will be implemented are:
\begin{itemize}
  \item Pac-Man's walking animation
  \item Pac-Man's death animation
  \item Ghost's movement
  \item Ghost's frightened
\end{itemize}

\textbf{Pause} The game can be paused using the "pause" or "esc" keys on the keyboard.

\textbf{Interaction with the file system} A running highscore will be kept in a file.

\section{Optional Requirements} \label{ch:optionalRequirements}

\textbf{Levels} Originally, Pac-Man only contains one level, that gets more difficult as you complete them. At least one additional level will be added as an option for the player.

\textbf{Different enemies} Pac-Man contains several ghosts. At a minimum Blinky will be implemented. Additional ghosts will be Pinky (pink), Inky (cyan) and Clyde (yellow).

\textbf{Multi-player} The original way to play against another player in Pac-Man is to beat eachother's highscore. Optionally, a way to control one of the ghosts will be added to the game.


\section{Pure/Impure seperation} \label{ch:pureSeperation}
\section{Abstraction} \label{ch:abstraction}








% \printbibliography

\end{document}


% \begin{lstlisting}[language=Haskell]
%   sum :: [a] -> a
%   sum = foldr (+) 0
% \end{lstlisting}

% \begin{Figure}
%   \centering
%   \includegraphics[scale=0.45]{placeholder.png}
%   \captionof{figure}{An example}
%   \label{fig:placeholder}
% \end{Figure}


% \begin{Figure}
%   \centering
%   \includegraphics[scale=0.45]{Bubble sensor diagram.png}
%   \captionof{figure}{Schematical overview of the hemispherical pressure sensor. [I] Silicone layer, [II] barometric sensor, [III] Working fluid, [IV] Adhesive, [V] Flex-PCB.}
%   \label{fig:SchematicalOverviewHemisphere}
% \end{Figure}
