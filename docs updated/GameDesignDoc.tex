\documentclass[11pt]{Article}
% \usepackage{fullpage}
\usepackage{graphicx}
\usepackage{amsmath}
\usepackage{pdfpages}
\usepackage{lipsum}
\usepackage{blindtext} 
\usepackage{fancyhdr}
\usepackage[colorinlistoftodos]{todonotes}


%syntax highlighting and code blocks
\usepackage{listings}
% \usepackage{minted}

\usepackage{tikz}
\usetikzlibrary{positioning,calc}

% Language setting
% Replace `english' with e.g. `spanish' to change the document language
% \usepackage[english]{babel}
\usepackage{multicol}
\usepackage{geometry}
\usepackage{caption}
\usepackage{float}
\usepackage[hidelinks]{hyperref}
\usepackage[backend=biber,style=apa,autocite=inline]{biblatex}
\DeclareLanguageMapping{english}{english-apa}
\setlength{\columnsep}{1cm}
\geometry{
        a4paper,
        total={170mm,257mm},
        left=20mm,
        top=20mm,
        }

\usepackage{listings}
\usepackage{xcolor}

%%%%%%%%%%%%%%%%%%%%%%%%%%%%%%%%%%%%%%%%%%%%%%%%%%%%%%%%%%%%%%%%%%%%%%%%%%%%%%%%%%%%%%%%%%%%%%%%
%%%%%%%%%%%%%%%%%%%%%%%%%%%%%%%%%%%%%%%%%%%%%%%%%%%%%%%%%%%%%%%%%%%%%%%%%%%%%%%%%%%%%%%%%%%%%%%%


\definecolor{codegreen}{rgb}{0,0.6,0}
\definecolor{codegray}{rgb}{0.5,0.5,0.5}
\definecolor{codepurple}{rgb}{0.58,0,0.82}
\definecolor{backcolour}{rgb}{0.95,0.95,0.92}

\lstdefinestyle{mystyle}{
    backgroundcolor=\color{backcolour},   
    commentstyle=\color{codegreen},
    keywordstyle=\color{blue},
    numberstyle=\tiny\color{codegray},
    stringstyle=\color{codepurple},
    basicstyle=\ttfamily\footnotesize,
    breakatwhitespace=false,         
    breaklines=true,                 
    captionpos=b,                    
    keepspaces=true,                 
    numbers=left,                    
    numbersep=5pt,                  
    showspaces=false,                
    showstringspaces=false,
    showtabs=false,                  
    tabsize=2
} 

%%%%%%%%%%%%%%%%%%%%%%%%%%%%%%%%%%%%%%%%%%%%%%%%%%%%%%%%%%%%%%%%%%%%%%%%%%%%%%%%%%%%%%%%%%%%%%%%
%%%%%%%%%%%%%%%%%%%%%%%%%%%%%%%%%%%%%%%%%%%%%%%%%%%%%%%%%%%%%%%%%%%%%%%%%%%%%%%%%%%%%%%%%%%%%%%%


% Useful packages
% \usepackage{amsmath}
% \usepackage{siunitx}
% \usepackage{wrapfig}
% \usepackage{float}
% \usepackage{graphicx}
% \usepackage{subcaption}
% \usepackage[colorlinks=true, allcolors=blue]{hyperref}
% \usepackage{xcolor}

\graphicspath{
    {img/}
}
\bibliography{./bib.bib}

\newenvironment{Figure}
  {\par\medskip\noindent\minipage{\linewidth}}
  {\endminipage\par\medskip}
\newcommand\C[1]\null

%remove if done, this extends the page to see comments
% \paperwidth=\dimexpr \paperwidth + 8cm\relax
% \oddsidemargin=\dimexpr\oddsidemargin + 3cm\relax
% \evensidemargin=\dimexpr\evensidemargin + 3cm\relax
% \marginparwidth=\dimexpr \marginparwidth + 3cm\relax


\title{Hac-man: Game design document}

\author{%
  \begin{tabular}{c c c}
    Mirko van de Hoef   &    Tom Selier\\
    \texttt{m.a.j.vandeHoef@students.uu.nl} & \texttt{t.b.j.selier@students.uu.nl} \\
    Student Number: 2176777 & Student Number: 5412498 
  \end{tabular}
}


\usepackage{eso-pic}
\usepackage{atbegshi}
\AtBeginShipoutFirst{\AddToShipoutPictureBG*{
    \begin{tikzpicture}[overlay, remember picture, outer sep=0pt, inner sep=1ex]
      \node [anchor=north west] at ($(current page.north)-(.5*\textwidth-1ex, \headheight)$)
      {\parbox{10cm}{UU University Utrecht\\ Functional Programming\\2024, Utrecht, Netherlands}};
    \end{tikzpicture}
}}




%%%%%%%%%%%%%%%%%%%%%%%%%%%%%%%%%%%%%%%%%%%%%%%%%%%%%%%%%%%%%%%%%%%%%%%%%%%%%%%%%%%%%%%%%%%%%%%%
%%%%%%%%%%%%%%%%%%%%%%%%%%%%%%%%%%%%%%%%%%%%%%%%%%%%%%%%%%%%%%%%%%%%%%%%%%%%%%%%%%%%%%%%%%%%%%%%
%beginning of document
\begin{document}
\lstset{style=mystyle}
\maketitle
% \begin{abstract}
% \lipsum[1]
% \end{abstract}
% {\bf \textit{Keywords--}} game game gamin
%%%%%%%%%%%%%%%%%%%%%%%%%%%%%%%%%%%%%%%%%%%%%%%%%%%%%%%%%%%%%%%%%%%%%%%%%%%%%%%%%%%%%%%%%%%%%%%%
%%%%%%%%%%%%%%%%%%%%%%%%%%%%%%%%%%%%%%%%%%%%%%%%%%%%%%%%%%%%%%%%%%%%%%%%%%%%%%%%%%%%%%%%%%%%%%%%

%introduction
\section{Introduction} \label{ch:Introduction}
The objective of this design document is to present the design choices for the creation of Pac-Man, along with their respective justifications. The decision to implement Pac-man is based on on the expected freedom the game offers in its implementation. Furthermore, Pac-Man will allow us to implement some of the optional requirements in addition to the minimum requirements. Additionally, some variations on the fundemental gameplay are introduced to make the game more unique.

\section{Data types} \label{ch:dataTypes}
%ghost and player

The first thing that were implemeted are models for the player and the ghosts. These objects need to move, and store some information about them. 

Movable objects in the game are called entities. These entities have a direction in which they are moving, a heading, a speed and a current position, stored in a record \textit{movement}. A distinction can be made twen heading and direction. Direction refers to the direction in which an entity is moving, whereas heading represents the direction in which the entity wants to move. For instance, Pacman moves up, but wants to move to the left, but there  is a wall in the way. In this example, the direction is up, and the heading is left.

Then, we can create a record for the player and the ghosts. A player gets lives and a score, next to being an entity. And the ghosts get some identifiers for it's name and it's current behaviour. The ghosts also hold a data type that reflects their current behaviour mode. This behaviour mode also holds the time for how long they should remain in that behaviour. Furthermore, the ghosts hold all target tilepositions for each different behaviourMode. A target tile is the tile which the ghost is attempting to move towards.

% add something about the animations

\begin{lstlisting}[language=Haskell]
  data Direction = Left | Right | Up | Down | Still deriving (Eq, Show)
  
  data GhostType = Inky | Pinky | Blinky | Clyde deriving (Eq, Show)
  
  type EntityPosition = (Float, Float)
  
  data Movement = MkMovement
    { direction :: Direction
    , speed     :: Float
    , position  :: EntityPosition
    , heading   :: Direction
    }
  
  type Time = Float
  
  data BehaviourMode = Chase Time | Scatter Time | Frightened Time| Home Time deriving (Show)
  


  instance Eq BehaviourMode where
    (==) :: BehaviourMode -> BehaviourMode -> Bool
    Chase _ == Chase _           = True
    Scatter _ == Scatter _       = True
    Frightened _ == Frightened _ = True
    Home _  == Home _            = True
    _ == _                       = False
  
  type Lives = Int
  
  data Animation = MkAnimation
    { frames     :: [Gloss.Picture]
    , index      :: Int
    , rate       :: Float
    , lastUpdate :: Float
    }
  
  -- Entity record that has all common types for each entity (ghost or player)
  data Entity = MkEntity
    { movement     :: Movement  -- holds all data regarding the movement
    , oldDirection :: Direction -- record of the old direction used to differntiate between heading and direction
    , animation    :: Maybe Animation
    }
  
  -- player record
  data Player = MkPlayer
    { entity :: Entity
    , lives  :: Lives
    , score  :: Int
    }
  
  -- ghost record
  data Ghost = MkGhost
    { entityG       :: Entity         -- all movement etc.
    , ghostName     :: GhostType      -- name of the ghost (used for patternmatch and debug)
    , behaviourMode :: BehaviourMode  -- current behaviour mode
    , targetTile    :: EntityPosition -- current target Tile
    , homeTile      :: EntityPosition -- respawn position
    , scatterCorner :: EntityPosition -- targetTile when behaviour is scatter
    , disAbleMove   :: Bool           -- disables the ghost
    , homeTime      :: Float          -- number of seconds the ghost stays in home after being eaten
    }
  
  instance Eq Ghost where
    (==) :: Ghost -> Ghost -> Bool
    g1 == g2 = ghostName g1 == ghostName g2
\end{lstlisting}

% levels
\noindent The maze consists of two type of tiles, where each tile is an x by x square in a grid. The tile can be either a floor or a wall where each floor can be either one of the consumbles or an empty floor. The walls can be either a corner or a straight wall, each with their own orientation. This will be implemented as follows:
\begin{lstlisting}[language=Haskell]
  data CornerOrientation = SE | NW | NE | SW deriving (Show)

  data WallOrientation = Horizontal | Vertical deriving (Show)
  
  data WallShape
    = MkCorner CornerOrientation
    | MkWallShape WallOrientation
    deriving (Show)
  
  data ConsumableType = Pellet | SuperPellet deriving (Show, Eq)
  
  data FloorType = MkConsumable ConsumableType | EmptyTile deriving (Show)
  
  data Tile = MkFloor FloorType | MkWall WallShape deriving (Show)
\end{lstlisting}


The maze will utilise a Map structure in order to speed up execution. This would result in a reduction in the scaling from $O(n)$ to $O(logn)$ for tile lookup. As the program is expected to lookup a tile every time one has to check for collision, it is crucial that this takes as little time as possible. This is only possible if the key, which is the x and y coordinates of the tile, can be ordered. The instance below is an example but is important to show that the keys are indeed orderable. In this case it is possible using the implementation below. Thus, two tiles in the maze looks like: [((1, 0), Floor (Consumable Pellet)), ((2, 0), Wall (Corner SE))]. 

\begin{lstlisting}[language=Haskell]
  type TilePosition = (Int, Int) -- (x, y)

  type Maze = Map TilePosition Tile
\end{lstlisting}

\newpage
%gamestate
The game state is the final data type that contains all the data of the game. It's job is to be the model in the MVC structure. 

The record entries that might not seem clear at first glance have been commented. It is important to make the destinction that "maze" contains a map with tile types, while "oldMaze" contains a picture with the most recently rendered maze.

Furthermore, it might seem strange that the resolution of the window is stored in the gamestate. However, the ga

\begin{lstlisting}[language=Haskell]
  type Name = String

  type Sprite = Gloss.Picture
  
  type Sprites = Map.Map Name Sprite
  
  data GameStatus = Running | GameOver | Paused | Quitting deriving (Eq, Show)
  
  newtype WindowInfo = MkWindowInfo {resolution :: (Int, Int)}
  
  type HighScores = [String]
  
  data GameState = MkGameState
    { status      :: GameStatus
    , maze        :: Maze
    , isNewMaze   :: Bool       -- boolean flag if the Maze has changed or not
    , oldMaze     :: Gloss.Picture
    , sprites     :: Sprites    -- map of sprites, used in animations
    , elapsedTime :: Float
    , deltaTime   :: Float
    , enableDebug :: Bool
    , windowInfo  :: WindowInfo -- resolution of window
    , player      :: Player
    , ghosts      :: [Ghost]
    , pelletC     :: (Int, Int) -- total vs eaten
    , highscores  :: HighScores -- list of highscores (retrieved from file)
    , level       :: Int        -- current level
    }
  
\end{lstlisting}


\section{Minumum Requirements} \label{ch:minumumRequirements}
\noindent \textbf{Player} The player can control Pac-Man using the WASD-keys on the keyboard.\\

\noindent \textbf{Enemies} The computer controls all of the four typical ghosts; Inky, Pinky, Blinky and Clyde.\\

\noindent \textbf{Randomness} A ghost can exist in three states; Scatter, chase and frightened. When a ghost is frightened it will pick a random direction at each intersection.\\

\noindent \textbf{Animation} The amount of animations in Pac-Man is limited. The animations are implemented are:
\begin{itemize}
  \item Pac-Man's walking animation
  \item Ghost's movement
  \item Ghost's frightened
\end{itemize}

\noindent \textbf{Pause} The game can be paused by p the "esc" key on the keyboard.\\

\noindent \textbf{Interaction with the file system} A running highscore is kept in a file. The top 10 highscores are kept, altough only the top is displayed.\\


\newpage
\section{Optional Requirements} \label{ch:optionalRequirements}

\textbf{Different enemies} Pac-Man contains several ghosts. Besides Blinky (red), additional ghosts are implemented; Pinky (pink), Inky (cyan) and Clyde (yellow). These ghosts all follow the traditional behaviour patterns. 

\textbf{Comples graphics} Pac-Man and the Ghosts all have animations which are loaded from the file-system.


\section{Pure/Impure seperation} \label{ch:pureSeperation}
In Haskell it's good practice to keep pure and impure as seperate as possible. In this game there's two major uses of the IO monad, image reading and highscore writing/reading. To avoid having the IO monad sprinkled throughout the codebase, these get loaded (using a bind) in the main function, then stored as pure data in the gamestate. 

Two example impure functions are

\begin{lstlisting}[language=Haskell]
  loadActiveSprites  :: IO [(Name, Sprite)]
  loadActiveSprites' ::    [(Name, String)] -> IO [Sprite]
\end{lstlisting}

These are used in Main.hs to load images into to the gamestate, using the function 

\begin{lstlisting}[language=Haskell]
  storeActiveSprites :: [(Name, Sprite)] -> GameState -> GameState
\end{lstlisting}

Note how the type signature of this function contains GameState $\rightarrow$ GameState, input to output. This is a common theme throughout the game (e.g., Gloss.Picture $\rightarrow$ Gloss.Picture). This is done so that functions are easy to pipeline together. An example is shown below, from Main.hs.

\begin{lstlisting}[language=Haskell]
  -- store the new info in state
  let
    state =
        ( storeActiveSprites sp
        . storePlayerAnimation playerAnimation
        . storeGhostAnimation [blinkyAnimation, pinkyAnimation, clydeAnimation, inkyAnimation]
        . loadHighScores highScoreContents
        ) initialState
\end{lstlisting}

\section{Abstraction} \label{ch:abstraction}
Several steps for abstraction have been taken. For example, the "moveStep" (In controller/EntityController.hs) function takes an entity, and does not care which entity (player or NPC) it is actually a part of. Because of the abstraction in the data types, this function can be used for the ghosts and player with no alteration. Furthermore, the animation record is identical for each entity, with the exception of the frames. 

Lastly, since tiles are both consumables or empty, it should be trivial to implement a function that updates a tile based on the position of the player. The function for this is given below, where the function arguments are the tile and player and state. The function than updates everything in the state based on the interaction between the tile and player.

\begin{lstlisting}[language=Haskell]
  handleConsumable :: GameState -> Player -> Tile -> GameState
  handleConsumable state player tile =
    case retrieveConsumable' tile of
      Just cType -> handleConsumable' state tilePos cType -- update score etc.
      _          -> state
   where
    tilePos = getTilePos $ (position . movement . entity) player




    
  handleConsumable' :: GameState -> TilePosition -> ConsumableType -> GameState
  handleConsumable' state@MkGameState{maze, player} pos cType =
    checkPelletCount (state
      { maze = Map.insert pos (MkFloor EmptyTile) maze --change tile to emptyTile
        ...
      }) (snd $ pelletC state)
  
\end{lstlisting}

\end{document}


% \begin{lstlisting}[language=Haskell]
%   sum :: [a] -> a
%   sum = foldr (+) 0
% \end{lstlisting}

% \begin{Figure}
%   \centering
%   \includegraphics[scale=0.45]{placeholder.png}
%   \captionof{figure}{An example}
%   \label{fig:placeholder}
% \end{Figure}


% \begin{Figure}
%   \centering
%   \includegraphics[scale=0.45]{Bubble sensor diagram.png}
%   \captionof{figure}{Schematical overview of the hemispherical pressure sensor. [I] Silicone layer, [II] barometric sensor, [III] Working fluid, [IV] Adhesive, [V] Flex-PCB.}
%   \label{fig:SchematicalOverviewHemisphere}
% \end{Figure}
